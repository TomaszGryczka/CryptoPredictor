\documentclass[a4paper,11pt]{article}

\usepackage[T1]{polski}
\usepackage[utf8]{inputenc}

\usepackage{listings}
\usepackage{hyperref}

\usepackage{float}
\usepackage{svg}

\lstset{
        basicstyle=\tt,
        literate={~}{{\fontfamily{ptm}\selectfont \textasciitilde}}1
}

\hoffset=-3.0cm
\textwidth=18cm
\evensidemargin=0pt

\voffset=-3cm
\textheight=27cm

\setlength{\parindent}{0pt}
\setlength{\parskip}{\medskipamount}
\raggedbottom

\newcommand\BS{\char`\\}
\newcommand\TY{\raise.17ex\hbox{$\scriptstyle\mathtt{\sim}$}}

\title{Projekt dyplomowy \\ Predykcja kursów na giełdzie kryptowalut z wykorzystaniem sztucznych sieci neuronowych}
\author{Tomasz Gryczka, 311341}


\begin{document}

\maketitle

\section{Cel projektu}
Celem projektu jest prognozowanie kursu jednej z najpopularniejszych kryptowalut - Bitcoin'a. W tym celu wykorzystane zostaną sztuczne sieci neuronowe. Pośrednim celem (jeśli starczy czasu) jest też porównanie dwóch typów sieci neuronowych: rekurencyjnych oraz transformerów. W trakcie pracy postaram też odpowiedzieć na pytanie, czy dzięki wykorzystaniu uczenia maszynowego jesteśmy w stanie osiągnąć lepsze wyniki w porównaniu do tradycyjnych metod inwestowania / spekulacji.
\section{Wykorzystane narzędzia}
Do wykonania powyższych zadań zostanie użyty język programowania \textbf{Python}. W obecnej chwili istnieją dwa najpopularniejsze frameworki przeznaczone dla sztucznych sieci neuronowych: TensorFlow i PyTorch. W trakcie pracy użyty zostanie \textbf{PyTorch}, który według Google Trends przebił już popularnością TensorFlow i wydaje się być bardziej przyszłościowym rozwiązaniem. Biblioteki \textbf{Pandas}, \textbf{Numpy} oraz \textbf{Scikit-learn} będą podstawowymi narzędziami pomocnymi w przetwarzaniu danych dotyczących historycznych kursów kryptowaluty i wygenerowanych prognoz.

Jak już zostało wcześniej wspomniane celem projektu jest porównanie dwóch typów sieci: rekurencyjnych (tutaj zastosowana zostanie sieć \textbf{LSTM}) oraz transformerów. Główny nacisk będzie kładziony na sieć rekurencyjną.

Wykorzystane zostaną ogólnodostępne źródła historycznych i aktualnych kursów kryptowaluty. Przykładowym źródłem są:
\begin{itemize}
  \item \href{https://finance.yahoo.com/quote/BTC-USD/history?period1=1410912000&period2=1679961600&interval=1d&filter=history&frequency=1d&includeAdjustedClose=true}{Yahoo! Finance},
  \item \href{https://www.investing.com/crypto/bitcoin/historical-data}{investing.com}.
\end{itemize}

W trakcie pracy prawdopodobnie zostanie wykorzystane środowisko Google Colab. Wszystkie efekty pracy będą trzymane w \href{https://github.com/TomaszGryczka/CryptoPredictor}{repozytorium GitHub}.

\end{document}